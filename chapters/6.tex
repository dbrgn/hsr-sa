% Chapter 6

\chapter{Development Process \& Design Decisions}

\label{ch:development}

%----------------------------------------------------------------------------------------

\section{Coding Guidelines}

\marginpar{The PEPs (Python Enhancement Proposals) are Python's way of
continuously improving the language in a community driven process. Any community
member can submit a proposal for a language change, which is then discussed and
accepted or rejected.}

Coding guidelines are important in order to achieve a consistent style
throughout the codebase. In the Python world, the PEP8 style guide
\cite{pep8:2001} has seen near ubiquitous adaptation and should be used for all
projects in order to aid the legibility of the source code.

\begin{quote}{\slshape
One of Guido's key insights is that code is read much more often than it is
written. The guidelines provided here are intended to improve the readability of
code and make it consistent across the wide spectrum of Python code. As PEP 20
says, ``Readability counts''. \\ \medskip
--- \defcitealias{pep8:2001}{Style Guide for Python Code}\citetalias{pep8:2001} \citep{pep8:2001}
}\end{quote}

\noindent \tangible{} follows most parts of PEP8, with two exceptions:

\begin{itemize}
	\item While line lengths below 80 characters are the ideal case, lines with up
		to 99 characters are still acceptable, if it makes the code more readable.
	\item Errors E126--E128, which specify indentation rules for multi-line
		statements, can be ignored.
\end{itemize}

%----------------------------------------------------------------------------------------

\section{Future Imports}

While Python 3 has first been released back in 2008, it has still not completely
managed to replace the older 2.x versions. As a result, the \tangible{} codebase
currently targets Python 2.7.

But in the past year several things have happened that accelerated the adoption
rate of Python 3. Most importantly, several big Linux distributions like Arch
Linux\footnote{\url{https://www.archlinux.org/}} and
Fedora\footnote{\url{http://fedoraproject.org/}} have decided to move to Python
3 as the default Python implementation. Another important factor was the newly
added Python 3 support in big Python frameworks like
Django\footnote{\url{https://www.djangoproject.com/}}.

In view of these facts, the \tangible{} source code is written in a forward
compatible way by adding ``future-imports'' to the top of every code file.
Python provides a module called \texttt{future} which contains backports of
newer language features to older Python versions. By using these imports, the
migration process to newer language versions can be simplified.

In the \tangible{} Project, the following preamble should be added to every
source code file:

\vspace{.5\baselineskip}
\begin{pythoncode}
# -*- coding: utf-8 -*-
from __future__ import print_function, division
from __future__ import absolute_import, unicode_literals
\end{pythoncode}

\noindent This results in the following effects:

\begin{itemize}
	\item The \texttt{\# -*- coding: utf-8 -*-} line tells the Python interpreter
		that this file is UTF8-encoded. In Python 3 UTF8 encoding will become the
		default.
	\item The \texttt{print\_function} import removes the \texttt{print} statement
		and adds a \texttt{print()} function.
	\item When activating the \texttt{division} import, division of two integers
		results in a \texttt{float} value instead of the old, lossy way of returning
		a floored integer. When floor division is explicitly desired, the
		\texttt{//} operator should be used instead.
	\item By importing \texttt{absolute\_import}, Python prioritizes absolute
		imports over relative imports. This fixes a few issues with import name
		clashes.
	\item The \texttt{unicode\_literals} import is the one with the most
		consequences of all the future imports listed above. It changes the default
		type of strings from bytestrings to unicode objects. By eliminating this
		Python 2/3 consistency from the beginning, many hard to spot migration bugs
		can be prevented.
\end{itemize}

\noindent The choice of future imports is based on the article \emph{Quick Tips
on Making Your Code Python 3 Ready} by Hristo Deshev \cite{deshev:2012}

\newpage
TODO: Code Generation Pattern

TODO: Coding guidelines

TODO: Testing
