% Chapter 1

\chapter{Motivation}

\label{ch:motivation} % For referencing the chapter elsewhere, use \autoref{ch:introduction} 

%----------------------------------------------------------------------------------------

Data visualization is definitely nothing new. Neither is software to do
statistical analysis or 3D model generation. And 3D printers have been around
since 1984. But what happens if all these things are combined? By doing that,
data visualization is taken to a physical level.

%----------------------------------------------------------------------------------------

\section{Data visualization}\label{sec:datavis}

\begin{quote}{\slshape
The main goal of data visualization is its ability to \textbf{visualize data},
communicating information clearly and effectivelty.} \\ \medskip
--- \defcitealias{friedman:2008}{Vitaly Friedman}\citetalias{friedman:2008} \citep{friedman:2008}
\end{quote}

Data visualization tries to make raw data more easily accessible. Changes in
datapoints over time should be visible, relations between different datasets
should become apparent, and at the same time the visualization should be easy to
understand and pleasant to look at.

The traditional means to visualize data were mostly two dimensional: Maps
visualize geographical and topological relations between objects and landmarks,
charts show a dataset in an easy to understand, graphical way and infographics
present complex information about a specific topic quickly and clearly.

With the advent of computers, data visualization became interactive. Data could
be visualized in two- and three dimensional ways, and by using different input
options of a computer a user could interact with the data and learn more about
it.

But digital 3D visualizations are still only two dimensional projections of
three dimensional objects. Data and its visualization can be taken to a new
level by making the visualizations tangible.

In the past, creating physical objects to convey information was not something
very common. It was mostly done by artists as a creative way to convey
information in the form of a sculpture or another type of object
\cite{day:2009}\cite{schenker:2012}.

%----------------------------------------------------------------------------------------

\section{The rise of affordable 3D printing}\label{sec:history-3dprinting}

Digital 3D representations of complex data have also been around for quite a
while\cite{marcus:2003}, but they were always confined to the digital world.
Mostly because it was impractical to convert a digital model to a physical
representation.

Industrial 3D printing and CNC milling have been available for about 3 decades,
but just until recently these machines were prohibitively expensive for regular
people that just wanted to visualize data. The only alternative was manual work.

\marginpar{The patent ``US5121329: Apparatus and method for creating
three-dimensional objects'' was granted to S. Scott Crump in 1992 and expired in
2009.}

During the last few years this changed. In 2009, US patent 5121329
\cite{us5121329:1992} expired, and with that prices for consumer-ready 3D
printers plummeted.

\begin{figure}[h]
	\centering
	\includegraphics[width=\textwidth]{images/US5121329-1.png}
	\caption{US Patent 5121329}
	\label{img:us5121329a}
\end{figure}

\marginpar{The RepRap project works on creating general-purpose self-replicating
machines capable of printing plastic objects, and making them freely available
for the benefit of everyone.}

The surge of new cheap 3D printers was largely due to an increasing number of
enthusiasts from the \emph{Hacker-} and \emph{Maker-}Communities that worked on
projects like the RepRap\footnote{\url{http://reprap.org/wiki/RepRap}}, a very
successful and influential project.

While many of these projects were started by communities as non commercial Open
Source\footnote{\url{http://opensource.org/}} and Open
Hardware\footnote{\url{http://www.oshwa.org/}} projects, new crowdfunding
platforms like Kickstarter\footnote{\url{http://www.kickstarter.com/}} and
Indiegogo\footnote{\url{http://www.indiegogo.com/}} made raising capital for new
3D printers feasable, something that would not have been possible 10 years ago.
