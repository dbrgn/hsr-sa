% Chapter 2

\chapter{Goals}

\label{ch:goals}

The goals of this thesis can be summarized as follows:

%----------------------------------------------------------------------------------------

\section{Features}

\begin{itemize}
	\item The result of the thesis is a Python library to visualize data as
		printable 3D objects. It should handle single- and multi-dimensional data.
	\item The library should provide a set of basic predefined shapes.
	\item It should be possible to create custom shapes using the provided
		primitives.
	\item The input and output should be decoupled. The library should act as a
		cross compiler. It should be possible to generate code for different
		backends.
	\item During the time of this thesis, the main targeted backend is
		OpenSCAD\footnote{\url{http://www.openscad.org/}}.
	\item The library should run on Python 2.7.
\end{itemize}

%----------------------------------------------------------------------------------------

\section{Usability}

\begin{itemize}
	\item The library should be
		pythonic\footnote{\url{http://stackoverflow.com/q/58968}} and easy to use.
	\item Comprehensive documentation should be available.
\end{itemize}

%----------------------------------------------------------------------------------------

\section{Quality}

\begin{itemize}
	\item The library should be well tested (at least 80\% test coverage).
	\item Tests should run automatically every time code is pushed to the
		repository.
	\item Change in test coverage should be measured each time the tests are run.
\end{itemize}

%----------------------------------------------------------------------------------------
