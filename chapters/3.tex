% Chapter 3

\chapter{Features {\&} Architecture}

\label{ch:features}

%----------------------------------------------------------------------------------------

\section{Overview}\label{sec:overview}

Tangible is implemented as a single Python package, without any external
dependencies.

The architecture of Tangible can be divided into four different parts: The
abstract syntax tree (AST), code generation backends, shapes and utils.

\subsection{AST}\label{sec:overview:ast}

The \texttt{ast.py} module provides the objects for the abstract syntax tree
(AST) for Tangible. It contains the following classes:

\subsubsection{Base class}

\begin{itemize}
	\item \texttt{AST}: The base shape for all AST elements.
\end{itemize}

\subsubsection{2D shapes}

\begin{itemize}
	\item \texttt{Circle}: A circle 2D shape.
	\item \texttt{CircleSector}: A circle sector (pizza slice).
	\item \texttt{Rectangle}: A rectangular 2D shape.
	\item \texttt{Polygon}: A polygon 2D shape.
\end{itemize}

\subsubsection{3D shapes}

\begin{itemize}
	\item \texttt{Cube}: A cube with a specified width, height and depth.
	\item \texttt{Sphere}: A sphere with a specified radius.
	\item \texttt{Cylinder}: A cylinder with a height and top/bottom radii.
	\item \texttt{Polyhedron}: An arbitrary 3D shape made from connected triangles
		or quads.
\end{itemize}

\subsubsection{Transformations}

\begin{itemize}
	\item \texttt{Translate}
	\item \texttt{Rotate}
	\item \texttt{Scale}
	\item \texttt{Mirror}
\end{itemize}

\subsubsection{Transformations}

\begin{itemize}
	\item \texttt{Union}
	\item \texttt{Difference}
	\item \texttt{Intersection}
\end{itemize}

\subsubsection{Extrusions}

\begin{itemize}
	\item \texttt{LinearExtrusion}
	\item \texttt{RotateExtrusion}
\end{itemize}

\subsection{Backends}\label{sec:overview:backends}

\subsection{Shapes}\label{sec:overview:shapes}

\subsection{Utils}\label{sec:overview:utils}

%----------------------------------------------------------------------------------------

\section{Features}\label{sec:features}

%----------------------------------------------------------------------------------------

\section{Usage}\label{sec:usage}

%----------------------------------------------------------------------------------------

\section{Future Possibilities}\label{sec:future}

%----------------------------------------------------------------------------------------
